\begin{abstract}
    Die Röntgengeräte der Firma Varian können Pixelfehler in den Bildern erkennen und korrigieren. Um dieses System in einem Testaufbau zu testen, braucht es einen Sensor, welcher deterministisch die zufallsbedingten Pixelfehler generiert und an die Algorithmen sendet. Einen solchen Emulator des Röntgensensors soll dieses Projekt entwickeln. Das Herzstück dieses Projektes ist ein Xilinx Zynq Ultrascale+ SoC. Ein Computer kann über USB 2.0 Daten auf das RAM des ARM-Systems schreiben und das ARM-System kann mit dem FPGA-System über AXI kommunizieren. 

\end{abstract}	
