\section{Imager-Emulator Konfiguration}
Die Konfigurationen können über USB auf den Imager-Emulator geladen werden. Die Daten sind im RAM in einer C-Struktur gespeichert. Im Moment hat die Struktur vier Einträge. Der \textbf{Enable}-Eintrag kann den Emulator stoppen. Die Idee ist, dass immer wenn der PC neue Konfigurationen und Bilder auf den Emulator lädt, dieses Bit zuerst gelöscht wird und somit der Emulator stoppt. Dies soll mögliche Fehler verhindern, welche wegen unvollständigen Konfigurationen und Bilder entstehen können. Sobald der Emulator vollständig konfiguriert ist, setzt der PC dieses Bit wieder. Der \textbf{n}-Eintrag gibt an wieviele Bilder auf dem Emulator gespeichert sind. Der \textbf{StartAddr}-Eintrag sagt, wo das erste Bild im RAM anfängt. Und der \textbf{Bildgrösse}-Eintrag sagt mit drei Bits welche Bildgrösse die Bilder haben. Somit kann das FPGA ausrechnen, wann das nächste Bild anfängt. Diese Konfigurationen sind an erster Stelle im ersten Sektor des USB-Massenspeichers. Das USB-Massenspeicher Protokoll definiert, dass immer ein Sektor aufs mal geschrieben wird. Ein Sektor ist 512 Byte gross. Sobald im ersten Sektor etwas geschrieben wird, aktualisiert das ARM-System die Register auf dem FPGA.

Um neue Konfigurationen hinzufügen muss lediglich die C-Struktur um die Konfiguration erweitert werden. Die Funktion, welche die Konfigurationen auf die FPGA-Register schreibt, kopiert dann diese Struktur. Zudem muss diese Struktur in der PC-Applikation nachgebildet sein, damit die Konfigurationen auch in der richtigen Reihenfolge auf den Massenspeicher geschrieben werden.